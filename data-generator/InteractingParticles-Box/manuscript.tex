\documentclass[letterpaper, 11pt]{article}

\usepackage{enumerate}
\usepackage{graphicx}
\usepackage{amsmath,amssymb,amsfonts}
\usepackage{amsthm}
\usepackage{mathrsfs}
\usepackage{xcolor}
\usepackage{mathtools}
\usepackage{hyperref}
\hypersetup{colorlinks=true,linkcolor=blue,urlcolor=blue,citecolor=blue}
\renewcommand{\tableautorefname}{Table}
\renewcommand{\figureautorefname}{Fig.}
\renewcommand{\sectionautorefname}{Sec.}
\renewcommand{\subsectionautorefname}{Sec.}
\renewcommand{\thetable}{\arabic{table}}

%%%%%%%%%%%%%%%%%%%%%%%%%%%%%%
% macros
\newcommand{\Set}[1]{\mathbb{#1}}
\global\long\def\Greens{\M{\Set G}}
\renewcommand{\i}{\mathsf{i}}
\global\long\def\V#1{\boldsymbol{#1}} %vector
\global\long\def\M#1{\boldsymbol{#1}} %matrix
\newcommand{\sM}[1]{\M{\mathcal{#1}}} %matrix in mathcal font
\newcommand{\tp}{^\intercal}
\newcommand{\ihat}{\V{e}_x}
\newcommand{\jhat}{\V{e}_y}


\global\long\def\norm#1{\left\Vert #1\right\Vert }
\global\long\def\abs#1{\left|#1\right|}

\global\long\def\grad{\M{\nabla}}
\global\long\def\av#1{\left\langle #1\right\rangle}

\newcommand{\kT}{k_B T}

\title{Point particles in a 2D box}

\begin{document}
\date{}
\maketitle
\thispagestyle{empty}
\section{Model}
\subsection{Governing equations}
We consider $N$ particles with radius $a$ in a two-dimensional box $x,y\in[0,L]$. The deterministic dynamics of the particles are governed by the Newton's second law
\begin{subequations}
\begin{align}
  m\frac{d\V{v}}{dt}&=-6\pi\eta a\V{v}+\left(\V{F}^{\text{ext}}(\V{x})+\V{F}^{\text{steric}}(\V{x})\right),\\
  \frac{d\V{x}}{dt}&=\V{v}.
\end{align}
\label{eq:equation-motion}%
\end{subequations}
Here the symbols stand for the velocity of the particles, $\V{v}=\begin{bmatrix}\V{v}_1,\V{v}_2,\dots\end{bmatrix}^{\tp}\in\Set{R}^{N\times 2}$; position of the particles, $\V{x}=\begin{bmatrix}\V{x}_1,\V{x}_2,\dots\end{bmatrix}^{\tp}\in\Set{R}^{N\times 2}$; vectors of, respectively, external, $\V{F}^{\text{ext}}(\V{x})\in\Set{R}^{N\times 2}$, and Lennard-Jones, $\V{F}^{\text{steric}}(\V{x})\in\Set{R}^{N\times 2}$, forces on the particles; viscosity, $\eta$; time, $t$.

In most practical cases, we would assume an overdamped regime, and drop the inertia term in \eqref{eq:equation-motion} and find a first-order differential equation for $\V{x}$:
\begin{equation}
  \frac{d\V{x}}{dt}=\mu\left(\V{F}^{\text{ext}}(\V{x})+\V{F}^{\text{steric}}(\V{x})\right),
  \label{eq:equation-motion-overdamped}
\end{equation}
where $\mu=1/(6\pi\eta a)$ is the mobility of the particles. But for this work, we consider the full system \ref{eq:equation-motion} and account for the acceleration as well. (This is to make it relevant to the available GNS code.)

We suppose that the external forces (e.g., due to an applied electric field) are known as functions of time and space. For the steric interactions, we consider a truncated Lennard-Jones (LJ) potential of the form
\begin{equation}
  U^{\text{steric}}(r)=
  \begin{cases}
    U_{\text{LJ}}(r_m)+\frac{dU_{\mathrm{LJ}}(r_m)}{dr}(r-r_m) & r\leq r_m,\\
    U_{\text{LJ}}(r) & r_m<r\leq R,\\
    0 & r>R,
  \end{cases}
  \label{Eq:LJPot}
\end{equation}
with
\begin{equation}
  U_{\text{LJ}}(r)=4U_0\left(\left(\frac{2a}{r}\right)^{2p}-\left(\frac{2a}{r}\right)^p\right).
  \label{Eq:LJPot}
\end{equation}
The symbols stand for the ion-ion distance, $r$; lower cutoff distance, $r_m$; upper cutoff distance, $R$; and depth of the potential well, $U_0$. The lower cutoff distance $r_m<2a$ determines the maximum repulsive force for a given set of parameters. By regulating the exponent $p$, one can have softer or harder interaction forces. The LJ force on particle $i$ due to particle $j$ is
\begin{align}
  \V{F}^{\text{steric}}_{ij}(\V{x})&=-\grad U^{\text{steric}}(r_{ij})=\nonumber\\
  &\begin{cases}
    -\frac{1}{r_{ij}}\frac{dU_{\mathrm{LJ}}(r_m)}{dr}\V{x}_{ij} & r_{ij}\leq r_m,\\
    \frac{4pU_0}{r_{ij}^2}\left(2\left(\frac{2a}{r_{ij}}\right)^{2p}-\left(\frac{2a}{r_{ij}}\right)^p\right)\V{x}_{ij} & r_m<r_{ij}\leq R,\\
    0 & r_{ij}>R,
  \end{cases}
  \label{Eq:LJForce}
\end{align}
with $\V{x}_{ij}=\begin{bmatrix}x_i-x_j & y_i-y_j\end{bmatrix}$ and $r_{ij}=\sqrt{x_{ij}^2+y_{ij}^2}$. Accordingly, the total LJ force experienced by particle $i$ is $\sum_{j\neq i}\V{F}^{\text{steric}}_{ij}(\V{x})$.

We cast the equations into a dimensionless form by using $a$, $\sqrt{a/g}$, and $mg$ as the length, time, and force scales, respectively. The dimensionless governing equations become
\begin{subequations}
  \begin{align}
    \frac{d\tilde{\V{v}}}{d\tilde{t}}&=-\tfrac{9}{2}\alpha\tilde{\V{v}}+\tilde{\V{F}}^{\text{ext}}(\tilde{\V{x}})+\tilde{\V{F}}^{\text{steric}}(\tilde{\V{x}}),\\
    \frac{d\tilde{\V{x}}}{d\tilde{t}}&=\tilde{\V{v}}.
  \end{align}
  \label{eq:equation-motion-dimensionless}%
\end{subequations}
where the decoration $\sim$ denotes a dimensionless variable and
\begin{equation}
  \alpha=\frac{\eta}{\rho_pa^{\tfrac{3}{2}}g^{\tfrac{1}{2}}}.
  \label{eq:dl-params}
\end{equation}
Similarly, the pair interaction force changes to
\begin{align}
  \tilde{\V{F}}^{\text{steric}}_{ij}(\tilde{\V{x}})&=\nonumber\\
  &\begin{cases}
    -\frac{1}{\tilde{r}_{ij}}\frac{d\tilde{U}_{\mathrm{LJ}}(\tilde{r}_m)}{d\tilde{r}}\tilde{\V{x}}_{ij} & \tilde{r}_{ij}\leq \tilde{r}_m,\\
    \frac{4p\tilde{U}_0}{\tilde{r}_{ij}^2}\left(2\left(\frac{2}{\tilde{r}_{ij}}\right)^{2p}-\left(\frac{2}{\tilde{r}_{ij}}\right)^p\right)\tilde{\V{x}}_{ij} & \tilde{r}_m<\tilde{r}_{ij}\leq \tilde{R},\\
    0 & \tilde{r}_{ij}>\tilde{R},
  \end{cases}
  \label{Eq:StericForce-dimensionless}
\end{align}
where,
\begin{equation}
  \frac{d\tilde{U}_{\mathrm{LJ}}(\tilde{r}_m)}{d\tilde{r}}=-\frac{4p\tilde{U}_0}{\tilde{r}_m}\left(2\left(\frac{2}{\tilde{r}_m}\right)^{2p}-\left(\frac{2}{\tilde{r}_m}\right)^p\right).
\end{equation}
Finally, $\tilde{U}_0=U_0/(mg)$ and $\tilde{r}_m=r_m/a$.

\subsection{Boundary conditions}
We impose solid-wall boundary conditions with an energy loss ratio $\gamma\in(0,1]$; i.e., upon each collision to a wall, the kinetic energy of the colliding particle is reduced to $\gamma$ multiplied to its pre-collision value. For simplicity we suppose that this energy loss occurs in both directions equally. Furthermore, the direction of the particle motion after collision is determined by assuming a mirror reflection from the wall. If we denote the velocity vectors before and after collision by $\V{v}$ and $\V{v}'$, respectively, then such a reflection can be expressed as
\begin{equation}
  \V{v}'=\sqrt{\gamma}\left[\V{v}-2\left(\V{v}\cdot\hat{\V{n}}\right)\hat{\V{n}}\right],
\end{equation}
where $\hat{\V{n}}$ is the normal unit vector pointing into the domain.

\subsection{Numerical Integration}
We use the Euler method to numerically integrate the equation of motion \eqref{eq:equation-motion-dimensionless}. Of course, we could use a more accurate numerical scheme such as RK4. But note that the main goal here is to generate some trajectories that can be used to train a GNS.

\end{document}
